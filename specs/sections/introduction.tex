\section{Introduction}
\subsection{Purpose}
By its nature, genomic data can include information of a confidential nature about the health of individuals.
%
It is important that such information is not accidentally disclosed.
%
One part of the defence against such disclosure is to, as much as possible, keep the data in an encrypted format.
%

This document describes a file format that can be used to store data in an encrypted and authenticated state.
%
Existing applications can, with minimal modification, read and write data in the encrypted format.
%
The choice of encryption also allows the encrypted data to be read starting from any location, facilitating indexed access to files.

\subsection{Requirements}
The keywords ``MUST'', ``MUST NOT'', ``REQUIRED'', ``SHALL'', ``SHALL NOT'', ``SHOULD'', ``SHOULD NOT'', ``RECOMMENDED'', ``MAY'', and ``OPTIONAL'' in this document are to be interpreted as described in~\cite{RFC2119}.

\subsection{Terminology}
%
\begin{description}
\item[Elliptic-curve cryptography (ECC)] %
  An approach to public-key cryptography based on the algebraic structure of elliptic curves over finite fields.
  % 
\item[Curve25519] %
  A widely used FIPS-140 approved ECC algorithm not encumbered by any patents~\cite{RFC7748}.
  % 
\item[Ed25519] %
  An Edwards-curve Digital Signature Algorithm (EdDSA) over Curve25519~\cite{RFC8032}.
  % 
\item[ChaCha20-IETF-Poly1305] %
  ChaCha20 is a symmetric stream cipher built on a pseudo-random function that gives the advantage that one can efficiently seek to any position in the key stream in constant time.
  % 
  It is not patented.
  % 
  Poly1305 is a cryptographic message authentication code (MAC).
  % 
  It can be used to verify the data integrity and the authenticity of a message~\cite{RFC8439}.
  % 
\item[ciphertext] %
  The encrypted version of the data.
  % 
\item[plaintext] %
  The unencrypted version of the data.
  %
\end{description}
