%%%%%%%%%%%%%%%%%%%%%%%%%%%%%%%%%%%%%%%%%%%%%%%%%%%%%%%%%%%%%
\section{Introduction}
%%%%%%%%%%%%%%%%%%%%%%%%%%%%%%%%%%%%%%%%%%%%%%%%%%%%%%%%%%%%%

By its nature, genomic data can include information of a confidential nature about the health of individuals.
%
It is important that such information is not accidentally disclosed.
%
One part of the defence against such disclosure is to, as much as possible, keep the data in an encrypted format.
%

This document describes a file format that can be used to store data in an encrypted and authenticated state.
%
Existing applications can, with minimal modification, read and write data in the encrypted format.
%
The choice of encryption also allows the encrypted data to be read starting from any location, facilitating indexed access to files.

%%%%%%%%%%%%%%%%%%%%%%%%%%%%%%%%%%%%%%%%%%%%%%%%%%%%%%%%%%%%%
\subsection{Encrypted Representation Overview}
%%%%%%%%%%%%%%%%%%%%%%%%%%%%%%%%%%%%%%%%%%%%%%%%%%%%%%%%%%%%%

The encrypted file consists of two parts: the header and the encrypted data portion.

The header encapsulates its data and prepends a magic number and a version number.
%
We describe its construction in Section~\ref{header}.

The encrypted data portion is the actual application data, described in Section~\ref{data:encrypted}.
% 
It is encrypted using a symmetric encryption algorithm, as specified in the header.
% 
The data is encrypted in 64 kilobytes segments. For each encrypted segment, a 12 bytes nonce is prepended and a 16 bytes MAC is appended.
% 
The encrypted data portion may include a checksum of the unencrypted data, if the checksum algorithm is listed in the header.


%%%%%%%%%%%%%%%%%%%%%%%%%%%%%%%%%%%%%%%%%%%%%%%%%%%%%%%%%%%%%
\subsection{Requirements}
%%%%%%%%%%%%%%%%%%%%%%%%%%%%%%%%%%%%%%%%%%%%%%%%%%%%%%%%%%%%%

The keywords ``MUST'', ``MUST NOT'', ``REQUIRED'', ``SHALL'', ``SHALL NOT'', ``SHOULD'', ``SHOULD NOT'', ``RECOMMENDED'', ``MAY'', and ``OPTIONAL'' in this document are to be interpreted as described in~\cite{RFC2119}.

%%%%%%%%%%%%%%%%%%%%%%%%%%%%%%%%%%%%%%%%%%%%%%%%%%%%%%%%%%%%%
\subsection{Terminology}
%%%%%%%%%%%%%%%%%%%%%%%%%%%%%%%%%%%%%%%%%%%%%%%%%%%%%%%%%%%%%

\begin{description}
\item[Elliptic-curve cryptography (ECC)] %
  An approach to public-key cryptography based on the algebraic structure of elliptic curves over finite fields.
  % 
\item[Curve25519] %
  A widely used FIPS-140 approved ECC algorithm not encumbered by any patents~\cite{RFC7748}.
  % 
\item[ChaCha20-IETF-Poly1305] %
  ChaCha20 is a symmetric stream cipher built on a pseudo-random function that gives the advantage that one can efficiently seek to any position in the key stream in constant time.
  % 
  It is not patented.
  % 
  Poly1305 is a cryptographic message authentication code (MAC).
  % 
  It can be used to verify the data integrity and the authenticity of a message~\cite{RFC8439}.
  % 
\item[ciphertext] %
  The encrypted version of the data.
  % 
\item[plaintext] %
  The unencrypted version of the data.
  %
\end{description}

%%%%%%%%%%%%%%%%%%%%%%%%%%%%%%%%%%%%%%%%%%%%%%%%%%%%%%%%%%%%%
\subsection{Notations}
%%%%%%%%%%%%%%%%%%%%%%%%%%%%%%%%%%%%%%%%%%%%%%%%%%%%%%%%%%%%%

Hexadecimal values are written using the digits 0-9, and letters A-F for values 10-15.
%
Values are written with the most-significant digit on the left, and prefixed with "0x".

The basic data size is the byte (8 bits).
%
All multi-byte values are stored in least-significant byte first (``little-endian'') order, called the byte ordering.
%
For example, the value 1234 decimal (0x4d2) is stored as the byte stream 0xd2 0x04.

Integers can be either signed or unsigned.
%
Signed values are stored in two's complement form.

\begin{center}
\begin{tabular}{l l l l}
\hline
\textbf{Name} & \textbf{Byte Ordering} & \textbf{Integer Type} & \textbf{Size (bytes)} \\
\hline
byte & & unsigned & 1 \\
le\_int32 & little-endian & signed & 4 \\
le\_uint32 & little-endian & unsigned & 4 \\
le\_int64 & little-endian & signed & 8 \\
le\_uint64 & little-endian & unsigned & 8 \\
le\_uint96 & little-endian & unsigned & 12 \\
\end{tabular}
\end{center}

Structure types may be defined (in C-like notation) for convenience.

\begin{verbatim}
struct demo {
  byte string[8];
  le_int32 number1;
  le_uint64 number2;
};
\end{verbatim}

When structures are serialized to a file, elements are written in the given order with no padding between them.
%
The above structure would be written as twenty bytes - eight for the array \kw{string}, four for the integer \kw{number1}, and eight for the integer \kw{number2}.

Enumerated types may only take one of a given set of values.
%
The data type used to store the enumerated value is given in angle brackets after the type name.
%
Every element of an enumerated type must be assigned a value.
%
It is not valid to compare values between two enumerated types.

\begin{verbatim}
enum Animal<le_uint32> {
  cat    = 1;
  dog    = 2;
  rabbit = 3;
};
\end{verbatim}

Parts of structures may vary depending on information available at the time of decoding.
%
Which variant to use is selected by an enumerated type.
%
There must be a case for every possible enumerated value.
%
Cases have limited fall-through.
%
Consecutive cases, with no fields in between, all contain the same fields.

\begin{verbatim}
struct AnimalFeatures {
  select (enum Animal) {
    case cat:
    case dog:
      le_uint32 hairyness;
      le_uint32 whisker_length;

    case rabbit:
      le_uint32 ear_length;
  };
};
\end{verbatim}

For the \kw{cat} and \kw{dog} cases, \kw{struct AnimalFeatures} is eight bytes long and contains two unsigned four-byte little-endian values.
%
For the \kw{rabbit} case it is four bytes long and contains a single four-byte little-endian value.

If the cases are different lengths (as above), then the size of the overall structure depends on the variant chosen.
%
There is no padding to make the cases the same length unless it is explicitly defined.
