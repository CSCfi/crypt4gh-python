%%%%%%%%%%%%%%%%%%%%%%%%%%%%%%%%%%%%%%%%%%%%%%%%%%%%%%%%%%%%%
\section{Detailed Specification}
%%%%%%%%%%%%%%%%%%%%%%%%%%%%%%%%%%%%%%%%%%%%%%%%%%%%%%%%%%%%%
%
%%%%%%%%%%%%%%%%%%%%%%%%%%%%%%%%%%%%%%%%%%%%%%%%%%%%%%%%%%%%%
\subsection{Overall Conventions}
%%%%%%%%%%%%%%%%%%%%%%%%%%%%%%%%%%%%%%%%%%%%%%%%%%%%%%%%%%%%%

Hexadecimal values are written using the digits 0-9, and letters A-F for values 10-15.
%
Values are written with the most-significant digit on the left, and prefixed with "0x".

The basic data size is the byte (8 bits).
%
All multi-byte values are stored in least-significant byte first (``little-endian'') order, called the byte ordering.
%
For example, the value 1234 decimal (0x4d2) is stored as the byte stream 0xd2 0x04.

Integers can be either signed or unsigned.
%
Signed values are stored in two's complement form.

\begin{center}
\begin{tabular}{l l l l}
\hline
\textbf{Name} & \textbf{Byte Ordering} & \textbf{Integer Type} & \textbf{Size (bytes)} \\
\hline
byte & & unsigned & 1 \\
le\_int32 & little-endian & signed & 4 \\
le\_uint32 & little-endian & unsigned & 4 \\
le\_int64 & little-endian & signed & 8 \\
le\_uint64 & little-endian & unsigned & 8 \\
le\_uint96 & little-endian & unsigned & 12 \\
\end{tabular}
\end{center}

Structure types may be defined (in C-like notation) for convenience.

\begin{verbatim}
struct demo {
  byte string[8];
  le_int32 number1;
  le_uint64 number2;
};
\end{verbatim}

When structures are serialized to a file, elements are written in the given order with no padding between them.
%
The above structure would be written as twenty bytes - eight for the array \kw{string}, four for the integer \kw{number1}, and eight for the integer \kw{number2}.

Enumerated types may only take one of a given set of values.
%
The data type used to store the enumerated value is given in angle brackets after the type name.
%
Every element of an enumerated type must be assigned a value.
%
It is not valid to compare values between two enumerated types.

\begin{verbatim}
enum Animal<le_uint32> {
  cat    = 1;
  dog    = 2;
  rabbit = 3;
};
\end{verbatim}

Parts of structures may vary depending on information available at the time of decoding.
%
Which variant to use is selected by an enumerated type.
%
There must be a case for every possible enumerated value.
%
Cases have limited fall-through.
%
Consecutive cases, with no fields in between, all contain the same fields.

\begin{verbatim}
struct AnimalFeatures {
  select (enum Animal) {
    case cat:
    case dog:
      le_uint32 hairyness;
      le_uint32 whisker_length;

    case rabbit:
      le_uint32 ear_length;
  };
};
\end{verbatim}

For the \kw{cat} and \kw{dog} cases, \kw{struct AnimalFeatures} is eight bytes long and contains two unsigned four-byte little-endian values.
%
For the \kw{rabbit} case it is four bytes long and contains a single four-byte little-endian value.

If the cases are different lengths (as above), then the size of the overall structure depends on the variant chosen.
%
There is no padding to make the cases the same length unless it is explicitly defined.

%%%%%%%%%%%%%%%%%%%%%%%%%%%%%%%%%%%%%%%%%%%%%%%%%%%%%%%%%%%%%
\subsection{Unencrypted Header}\label{unencrypted:header}
%%%%%%%%%%%%%%%%%%%%%%%%%%%%%%%%%%%%%%%%%%%%%%%%%%%%%%%%%%%%%

The file starts with an unencrypted header, with the following structure:

\begin{verbatim}
struct Unencrypted_header {
  byte      magic_number[8];
  le_uint32 version;
  le_uint32 header_len;
};
\end{verbatim}

The \kw{magic\_number} is the ASCII representation of the string ``crypt4gh''.

The version number is stored as a four-byte little-endian unsigned integer.
%
The current version number is 1.

\kw{header\_len} is the length of the \emph{remainder} of the header, stored as a four-byte little-endian unsigned integer.
%
The remainder of the header, thereafter called the \emph{header data} includes the public key, the nonce, the encrypted header data, and an authentication tag.
%

The current byte representation of the magic number and version is:
\begin{verbatim}
0x63 0x72 0x79 0x70 0x74 0x34 0x67 0x68 0x01 0x00 0x00 0x00
============= magic_number============= ===== version =====
\end{verbatim}

The following is an example of a header configuration:
%
\begin{verbatim}
[magic number][version][header len][signing key][nonce][encrypted header][mac]
\end{verbatim}


%%%%%%%%%%%%%%%%%%%%%%%%%%%%%%%%%%%%%%%%%%%%%%%%%%%%%%%%%%%%%
\subsection{Encrypted Header}\label{encrypted:header}
%%%%%%%%%%%%%%%%%%%%%%%%%%%%%%%%%%%%%%%%%%%%%%%%%%%%%%%%%%%%%
%

The plaintext header has is encoded using the type \kw{EncryptionParameters}, as follows:

\begin{verbatim}
enum EncryptionMethod<le_uint32> {
  chacha20_ietf_poly1305 = 0;
};

enum ChecksumAlgorithm<le_uint32> {
  none = 0;
  md5 = 1;
  sha256 = 2;
};


struct EncryptionParameters {
  enum ChecksumAlgorithm<le_uint32> checksum_algorithm;

  enum EncryptionMethod<le_uint32> method;
  select (method) {
    case chacha20_ietf_poly1305:
      byte       key[32];
  };
};
\end{verbatim}

% ---------------
\kw{method} is an enumerated type that describes the type of encryption to be used.

\kw{key} is a secret encryption key.
%
In the case of \kw{chacha20\_ietf\_poly1305}, it is treated as a concatenation of eight 32-bit little-endian integers.

% ---------------
\kw{checksum\_algorithm} is an enumerated type that describes the algorithm used for the checksum to be used for the \emph{unencryted} data content.
%
If a checksum algorithm is chosen, `sha256' SHOULD be prefered and `md5' SHOULD only be used for backwards compatibility.
%
Moreover, the checksum value is of the following form and appended at the end of the encrypted data portion (see Section~\ref{encrypted:data}). Nothing is appended if \kw{checksum\_algorithm} is \kw{none}.
%
\begin{verbatim}
select (checksum_algorithm) {
  case md5:
    byte       key[16];
  case sha256:
    byte       key[32];
  };
\end{verbatim}


% ---------------------------------
Finally, the encrypted header is generated by encoding the plaintext header in the Curve25519 message format~\cite{RFC7748}.
%
The public key and nonce, used in the encryption, are prepended, and the Poly1305 authentication tag is appended.

%%%%%%%%%%%%%%%%%%%%%%%%%%%%%%%%%%%%%%%%%%%%%%%%%%%%%%%%%%%%%
\subsection{Encrypted Data}\label{encrypted:data}
%%%%%%%%%%%%%%%%%%%%%%%%%%%%%%%%%%%%%%%%%%%%%%%%%%%%%%%%%%%%%
\subsubsection{ChaCha20-Poly1305 Encryption}

Informally, ChaCha20 works like a block cipher over blocks of 64 bytes.
%
It is initialized with a 256-bit key, a nonce, and a counter.
%
It produces a succession of blocks of 64 bytes, where the counter is incremented by one for each successive block, forming a keystream.
%
The counter usually starts at 1.
%
The ciphertext is the message combined with the output of the keystream using the XOR operation.
%
The ciphertext does not include any authentication tag.
%
In IETF mode, the nonce is 96 bits long and the counter is 32 bits long.

ChaCha20-Poly1305 uses ChaCha20 to encrypt a message and uses the Poly1305 algorithm to generate a 16-byte MAC over the ciphertext, which is appended at the end of the ciphertext.
%
The MAC is generated for the whole ciphertext that is provided, and appended to the ciphertext.
%
It is not possible to authenticate partially decrypted data.

% ------------------------
\subsubsection{Segmenting the input}
%
While ChaCha20 allows to decrypt individual blocks (using the appropriate nonce and counter values), the authentication tag is calculated over the whole ciphertext.
%
To retain streaming and random access capabilities, it is necessary to ensure that segments of the data can be authenticated, without having to read and process the whole file or stream.
%
In this format, the plaintext is divided into 64 kilobytes segments, and each segment is encrypted using ChaCha20-Poly1305, and a randomly-generated 96-bit nonce. The last segment must be smaller.
%
The nonce is prepended to the encrypted segment. Recall that ChaCha20-Poly1305 appends a 16-bytes MAC to the ciphertext.
%
This expands the data by 28 bytes, so a 65536 byte plaintext input will become a 65564 byte encrypted and authenticated ciphertext output.

% ------------------------
\subsubsection{Decryption}
The plaintext is obtained by authenticating and decrypting the encrypted segment(s) enclosing the requested byte range $[P;Q]$, where $P<Q$.
%
For a range starting at position $P$, the location of the segment $\text{seg\_start}$ containing that position must first be found.
%
$$\text{seg\_start} = \text{header\_len} + \text{floor}(P/65536) * 65564$$
%
For an encrypted segment starting at position $\text{seg\_start}$, 12-bytes are read to obtain the nonce, then the 65564 bytes of ciphertext (possibly fewer of it was the last segment), and finally the 16 bytes MAC.

An authentication tag is calculated over the ciphertext from that segment, and bitwise compared to the MAC. The ciphertext is authenticated if and only if the tags match.
%
An error should be reported if the ciphertext is not authenticated.

The key and nonce are used to produce a keystream, using ChaCha20 as above, and combined with the ciphertext using the XOR function to obtain the plaintext segment.

Successive segments are decrypted, until the last segment for the range $[P;Q]$ starting at position $\text{seg\_end}$, where %
%
$$\text{seg\_end} = \text{header\_len} + \text{floor}(Q/65536) * 65564$$
%
Plaintext segments are concatenated to form the resulting output, granted that $P \mathbin{\%} 65536$ bytes are discarded from the beginning of the first segment, and only $Q \mathbin{\%} 65536$ bytes are retained from the last segment.

Implementation details for ChaCha20-Poly1305 (ietf mode) are described in \cite{RFC8439}.

% ------------------------
\paragraph{Message digest over the unencrypted data}%
Finally, recall that the plaintext header stipulates which checksum algorithm was used to compute the message digest over the unencrypted data. If the chosen algorithm was not \kw{none}, the bytes representing the checksum value are discarded from the end of the resulting output to the recover the plaintext data.
% 
Naturally, the checksum value is bitwise compared to a newly computated message digest over the plaintext data. An error MUST be reported if the values do not match.

It is not possible to compare checksums when decrypting a given range of the file. It is only used when decrypting the entire file.
