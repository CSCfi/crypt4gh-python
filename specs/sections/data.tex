%%%%%%%%%%%%%%%%%%%%%%%%%%%%%%%%%%%%%%%%%%%%%%%%%%%%%%%%%%%%%
\section{Encrypted Data}\label{data:encrypted}
%%%%%%%%%%%%%%%%%%%%%%%%%%%%%%%%%%%%%%%%%%%%%%%%%%%%%%%%%%%%%
\subsection{ChaCha20-Poly1305 Encryption}\label{data:encrypted:chacha20poly1305}

Informally, ChaCha20 works like a block cipher over blocks of 64 bytes.
%
It is initialized with a 256-bit key, a nonce, and a counter.
%
It produces a succession of blocks of 64 bytes, where the counter is incremented by one for each successive block, forming a keystream.
%
The counter usually starts at 1.
%
The ciphertext is the message combined with the output of the keystream using the XOR operation.
%
The ciphertext does not include any authentication tag.
%
In IETF mode, the nonce is 96 bits long and the counter is 32 bits long.

ChaCha20-Poly1305 uses ChaCha20 to encrypt a message and uses the Poly1305 algorithm to generate a 16-byte MAC over the ciphertext, which is appended at the end of the ciphertext.
%
The MAC is generated for the whole ciphertext that is provided, and appended to the ciphertext.
%
It is not possible to authenticate partially decrypted data.

% ------------------------
\subsection{Segmenting the input}
%
While ChaCha20 allows to decrypt individual blocks (using the appropriate nonce and counter values), the authentication tag is calculated over the whole ciphertext.
%
To retain streaming and random access capabilities, it is necessary to ensure that segments of the data can be authenticated, without having to read and process the whole file or stream.
%
In this format, the plaintext is divided into 64 kilobytes segments, and each segment is encrypted using ChaCha20-Poly1305, and a randomly-generated 96-bit nonce. The last segment must be smaller.
%
The nonce is prepended to the encrypted segment. Recall that ChaCha20-Poly1305 appends a 16-bytes MAC to the ciphertext.
%
This expands the data by 28 bytes, so a 65536 byte plaintext input will become a 65564 byte encrypted and authenticated ciphertext output.

% ------------------------
\subsection{Decryption}
The plaintext is obtained by authenticating and decrypting the encrypted segment(s) enclosing the requested byte range $[P;Q]$, where $P<Q$.
%
For a range starting at position $P$, the location of the segment $\text{seg\_start}$ containing that position must first be found.
%
$$\text{seg\_start} = \text{header\_len} + \text{floor}(P/65536) * 65564$$
%
For an encrypted segment starting at position $\text{seg\_start}$, 12-bytes are read to obtain the nonce, then the 65564 bytes of ciphertext (possibly fewer of it was the last segment), and finally the 16 bytes MAC.

An authentication tag is calculated over the ciphertext from that segment, and bitwise compared to the MAC. The ciphertext is authenticated if and only if the tags match.
%
An error should be reported if the ciphertext is not authenticated.

The key and nonce are used to produce a keystream, using ChaCha20 as above, and combined with the ciphertext using the XOR function to obtain the plaintext segment.

Successive segments are decrypted, until the last segment for the range $[P;Q]$ starting at position $\text{seg\_end}$, where %
%
$$\text{seg\_end} = \text{header\_len} + \text{floor}(Q/65536) * 65564$$
%
Plaintext segments are concatenated to form the resulting output, granted that $P \mathbin{\%} 65536$ bytes are discarded from the beginning of the first segment, and only $Q \mathbin{\%} 65536$ bytes are retained from the last segment.

Implementation details for ChaCha20-Poly1305 (ietf mode) are described in \cite{RFC8439}.

% ------------------------
\paragraph{Message digest over the unencrypted data}%
Finally, recall that the plaintext header stipulates which checksum algorithm was used to compute the message digest over the unencrypted data. If the chosen algorithm was not \kw{none}, the bytes representing the checksum value are discarded from the end of the resulting output to the recover the plaintext data.
% 
Naturally, the checksum value is bitwise compared to a newly computated message digest over the plaintext data. An error MUST be reported if the values do not match.

It is not possible to compare checksums when decrypting a given range of the file. It is only used when decrypting the entire file.
